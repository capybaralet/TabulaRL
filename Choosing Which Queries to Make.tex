\documentclass{article}
\usepackage[utf8]{inputenc}
\usepackage{natbib}
\usepackage{graphicx}
\usepackage{amsmath}
\usepackage{amsfonts}
\usepackage{mathtools}


\title{Choosing Which Queries to Make}
\author{David Krueger, John Salvatier}
\date{August 2016} 
\begin{document}
\maketitle




We're proposing a few methods for how to choose the number of queries per state-action in model-based active RL.
We consider the tabular MDP case, with state and actions sets $\mathcal{S}, \mathcal{A}$, and transition and reward functions $\mathcal{T}, \mathcal{R}$.

We suppose each query has a fixed cost of $c$, and let $\mathcal{Q} := \mathbb{N}^{| \mathcal{S} \times \mathcal{A} |}$ represent the set of all possible sets of queries.
Then let $q_t \in \mathcal{Q}$ be the set of queries which have been performed at time $t$.
We can make our decision to query (or not) by specifying some \emph{desired query set}, $q^{max}_t \in \mathcal{Q}$, representing to set of all queries we'd like to make, and querying whenever we have not yet made all of those queries, i.e. if $q_t < q^{max}_t$.

Our baseline heuristic considers a fixed desired query set, $q^{max}_t = (n,...,n)$, with $n$ as a hyperparameter.
It seems clear that the optimal $n \rightarrow \infty$ as $c \rightarrow 0$, or $\gamma \rightarrow 1$, and $n \rightarrow 0$ as $c \rightarrow \infty$ or $\gamma \rightarrow 0$.
But it's not clear what specific value of $n$ to choose.
Since we care about our cost during learning, we'd like a way to choose $n$ before learning (or adjusting it during learning).

% Simulated Query Rollouts
% TODO
\section{A Proposal for Tuning $n$}
We propose tuning $n$ via Simulated Query Rollouts (SQR). 

The basic idea is to pick $n$, sample $r$ several times to compute how well PSRL does with that $n$. 
  
\begin{enumerate}
    \item For $i < O$
    \item Sample $\{\tilde{r}\} \sim \tilde{\mathcal{R}}$.
    \item For each query number $n < N $.
    \item Simulate PSRL using $\{\tilde{r}\}$ as the ground truth.
    \item compute actual performance from the discounted returns minus discounted query costs. $ P_{n,i} = \sum_{t=0} \gamma^i (r_t - \mathbb{I}[q_{t,s,a} < n]c)  $.
    \item Then use $n = \mathop{\mathrm{argmax}}\limits_n \mathbb{E}({P_n}) $
\end{enumerate}

The running time to compute $n$ here is roughly $N * O * \mathop{PSRL}(S, A, T) $

As $O$ gets large, the expectation of the performance should be accurate and $n$ will be optimal. Of course the computation is relatively inefficient. 

% Approximate Simulated Query Rollouts
\section{A Cheaper Approximation}
We now present a procedure for Approximate Simulated Query Rollout (ASQR).
To remove the need to run PSRL at step 3 in the loop of SQR, we can instead:

\begin{enumerate}
\item Simulate all of the queries in the desired query set via sampling the necessary $\{\tilde{r}\} \sim \tilde{\mathcal{R}}$.
\item Update $P_{agent} (\mathcal{R} | \{\tilde{r}\})$ using Bayes theorem.
\item Do planning and compute expected performance using the expected rewards under this updated posterior (and the cost of all the simulated queries).
\end{enumerate}

This only involves one iteration of planning, unlike PSRL, which performs planning numerous times in the loop.
This approach may under-estimate performance because the actual PSRL agent might not end up making all of the desired queries (and hence pay less query cost).
Or it may over-estimate performance, since the PSRL agent does not have the benefit of observing the queries up-front, and must make decisions with less information during learning.
These differences would be especially large when many states or important states are not very reachable (and so would be queried more slowly or less often by the PSRL agent).

\section{Further Extensions}
We can run the above algorithms at any time during learning, in order to dynamically adjust our desired query set based on new information we've acquired.
At a high level, this is analogous to the way Thompson Sampling resamples throughout learning.

Since querying each state equally often is likely highly suboptimal, we'd like to consider a broader class of desired query sets, but the number of such sets grows exponentially in the total number of queries.
We believe techniques for linear bandits might be used to specify the priority of each possible query at a given moment, thus specifying a strict total order over desired query sets (with the smallest element being the queries already performed).
Then we could use the above techniques to instead tune which index in this order to select as $q^{max}_t$ at any given $t$.

The connection with linear bandits comes from viewing each state, action pair, $(s,a)$, as one dimension of arm-space.
Note that given transition operator $\mathcal{T}$, each policy, $\pi$ induces a distribution over $(s,a)$, and hence a point on the $L_1$ sphere in arm-space.
Unlike a standard linear bandit, however, our queries are restricted to standard basis vectors, while only the arms corresponding to valid policies can ultimately be exploited.
Furthermore, an agent must visit a state, action in order to query it (although notably, this is \emph{not} the case for ASQR.




%%%%%%%%%%%%%%%%%%%%%%%%%%%%%%%%%%%%%%%%%%%%%%%%%%%%%%%%%%%%%%%%%%%%%%%%%%%%%%%%%%%%%%%%%%%%%%%%%%%%%%%%%%%%%%%%%%%%%%%%%%%

\bibliographystyle{plainnat}
%\bibliographystyle{plain}
\bibliography{references}
\end{document}
